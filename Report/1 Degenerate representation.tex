\documentclass[10pt,a4paper]{article}
\usepackage[top=2.3cm,bottom=1.2cm,left=1.5cm,right=1.5cm]{geometry}
\usepackage[utf8]{inputenc} 
\usepackage[T1]{fontenc}   
\usepackage{indentfirst}
\setlength{\parindent}{2em}
\usepackage{multicol}
\usepackage{setspace}
\usepackage{bm}
\usepackage{pifont}
\usepackage{float}
\usepackage{amssymb}
\usepackage[tbtags]{amsmath} 
\usepackage{tikz}

\usepackage[
    backend=biber,      
    style=numeric-comp, 
    sorting=none        
]{biblatex}
\addbibresource{references.bib} 
\usepackage{hyperref}

\numberwithin{equation}{section}
\renewcommand{\baselinestretch}{1.5}

\newcommand{\ket}[1]{\left| #1 \right\rangle}
\newcommand{\vev}[1]{\left\langle #1 \right\rangle}


\begin{document}
\title{Degenerate representation of $\mathfrak{sl}_{2}$ WZW}
\maketitle

\tableofcontents

\section{Affine Symmetry}
\subsection{Symmetry algebra}
The $G$ Wess-Zumino-Witten model is a conformal field theory with a $\hat{\mathfrak{g}}$ symmetry algebra. We define a number dim$\mathfrak{g}$ of 
holomorphic $\hat{\mathfrak{g}}$ currents $J^{a}(z)$ through their OPEs,
\begin{equation}
    \boxed{
        J^{a}(z)J^{b}(w) = \frac{kK^{ab}}{(z-w)^{2}} + \frac{f^{ab}_{c} J^{c}(w)}{z-w} + \mathcal{O}(1), \label{OPE}
        }
\end{equation}
where the constant $k$ is called the level, $f^{ab}_{c}$ are the structure constants of Lie algebra $\mathfrak{g}$ and 
$K^{ab} = \frac{1}{2g} f^{ac}_{d}f^{bd}_{c}$ is the Killing form. Here $g$ is the dual Coxeter number of $\mathfrak{g}$. In the case of 
$\mathfrak{sl}_{2}$, $g = 2$.\\
\par The modes of the current are defined by 
\begin{equation}
    J^{a}_{n} = \oint \mathrm{d}z \, z^{n} J^{a}(z).
\end{equation}
From the OPEs \eqref{OPE}, we deduce the following commutation relations
\begin{equation}
    \left[ J^{a}_{m}, J^{b}_{n} \right] = f^{ab}_{c} J^{c}_{m+n} + m k K^{ab} \delta_{m+n,0}. \label{CR1}
\end{equation}
We introduce the Sugawara construction for the the energy momentum tensor $T(z)$
\begin{equation}
    \boxed{T(z) = \frac{K_{ab} : J^{a}(z) J^{b}(z) : }{2(k+g)}.} \label{EM}
\end{equation}
$T$ is a Virasoro field of central charge $c = \frac{k \mathrm{dim} \mathfrak{g}}{k+g}$. The corresponding Virasoro generators are given by
\begin{equation}
    L_{n} = \frac{K_{ab}}{2(k+g)} : \sum_{m \in \mathbb{Z} } J^{a}_{n-m} J^{b}_{m} :.
\end{equation}
Using the commutation relations for the $J^{a}_{n}$ \eqref{CR1}, we can deduce commutation relation between $L_{m}$ and $J^{a}_{n}$:
\begin{equation}
    \left[ L_{m}, J^{a}_{n} \right] = -n J^{a}_{m+n}. \label{CR2}
\end{equation}

\subsection{Irreducible representations of \texorpdfstring{$\mathfrak{sl}_{2}$}{Lg}}
For simplicity, we will work in the m-basis, where the Casimir operator $C=K_{ab}J^{a}_{0}J^{b}_{0}$ and $J^{0}_{0}$ are diagonalized. 
A state is labeled as $\ket{j,m}$. The eigenvalue of the Casimir operator is $2j(j+1)$.\\
\par The action of $\mathfrak{sl}_{2}$ generators on state $\ket{j,m}$ is 
\begin{equation}
    \left\{
        \begin{aligned}
            J^{+}_{0} \ket{j,m} & = (j-m) \ket{j,m+1}\\
            J^{0}_{0} \ket{j,m} & = m \ket{j,m}\\
            J^{-}_{0} \ket{j,m} & = (j+m) \ket{j,m-1}
        \end{aligned}
    \right.
\end{equation}
Especially we find 
\begin{equation}
    J^{\pm}_{0} \ket{j,\pm j} = 0.
\end{equation}
Hence we can classify irreps by the existence of highest/lowest weight states $\ket{j,\pm j}$\cite{Ribault:2014hia}. The principle series representations
$\mathcal{C}^{j}_{\alpha}$ contain neither of these two, while the discrete series representations $\mathcal{D}^{j,\pm}$ contain
$\ket{j,\mp j}$ respectively. The finite-dimensional representations contain both of these two states. The eigenvalues of $J^{0}_{0}$ in these 
representations are listed below:

\begin{center}
    \begin{tabular}{|l|l|l|}
        \hline
        Representations&Parameter values&Eigenvalues of $J^{0}_{0}$\\
        \hline
        $\mathcal{C}^{j}_{\alpha}$ & $j \in -\frac{1}{2} + i \mathbb{R} $, $\alpha \in \mathbb{R}$ mod $\mathbb{Z}$& $\alpha + \mathbb{Z}$\\
        $\mathcal{D}^{j,+}$ & $j \in (-\infty,-\frac{1}{2}) $& $-j + \mathbb{N}$\\
        $\mathcal{D}^{j,-}$ & $j \in (-\infty,-\frac{1}{2}) $& $j - \mathbb{N}$\\
        \hline
    \end{tabular}
\end{center}


\section{Degenerate Representations of \texorpdfstring{$\widehat{\mathfrak{sl}_{2}}$}{Lg}}
The $\widehat{\mathfrak{sl}_{2}}$ degenerate representations can be derived from the KZ-BPZ relation. The degenerate representations $V^{\vev{r,s}}$
 are labeled by two integer $r$ and $s$, which has spin
\begin{equation}
    j_{\vev{r,s}} = \frac{s-1}{2} - \frac{k+2}{2} r \quad \mathrm{for} \quad s\geq 1, r \geq 0,
\end{equation}
and contains null vectors of level $N=rs$.
\subsection{Level 1 null vector}
The only possibility to have a level 1 null vector is $r = s = 1$. The corresponding null vector is given by \cite{Stocco:2022gah}:
\begin{equation}
    \hat{N}^{c}_{-1} = K_{ab} J^{a}_{-1} J^{b}_{0} J^{c}_{0} + j_{1,1} f^{c}_{ab} J^{a}_{-1} J^{b}_{0} - 2 j^{2}_{1,1} J^{c}_{-1}. \label{NV}
\end{equation}
The set of states $\left\{ \hat{N}^{a}_{-1} \ket{j,m} \right\}$ forms a subrepresentation of the degenerate representation.\\
\par The commutation relations of $J^{a}_{0}$ and $J^{a}_{1}$ with null vector $\hat{N}^{c}_{-1}$ are:
\begin{equation}
    \begin{aligned}
        \left[ J^{a}_{0}, \hat{N}^{c}_{-1} \right] 
        =& K_{db} \left[J^{a}_{0}, J^{d}_{-1}J^{b}_{0}J^{c}_{0} \right] + 
            j_{1,1}f^{c}_{db}\left[J^{a}_{0},J^{d}_{-1}J^{b}_{0}\right] - 2 j^{2}_{1,1} \left[J^{a}_{0},J^{c}_{-1}\right]\\
        =& K_{db} \left(f^{ad}_{e}J^{e}_{-1}J^{b}_{0}J^{c}_{0} + f^{ab}_{e}J^{d}_{-1}J^{e}_{0}J^{c}_{0} + f^{ac}_{e}J^{d}_{-1}J^{b}_{0}J^{e}_{0}\right)\\
            &+ j_{1,1}f^{c}_{db} \left( f^{ad}_{e} J^{e}_{-1}J^{b}_{0} + f^{ab}_{e} J^{d}_{-1} J^{e}_{0} \right)\\
            &-2 j^{2}_{1,1} f^{ac}_{e} J^{e}_{-1}\\
        =& f^{ac}_{e} \hat{N}^{e}_{-1}
    \end{aligned}
\end{equation}

\begin{equation}
    \begin{aligned}
        \left[ J^{a}_{1}, \hat{N}^{c}_{-1} \right] 
        =& K_{db} \left[J^{a}_{1}, J^{d}_{-1}J^{b}_{0}J^{c}_{0} \right] + 
            j_{1,1}f^{c}_{db}\left[J^{a}_{1},J^{d}_{-1}J^{b}_{0}\right] - 2 j^{2}_{1,1} \left[J^{a}_{1},J^{c}_{-1}\right]\\
        =& K_{db} \left(f^{ad}_{e}J^{e}_{0}J^{b}_{0}J^{c}_{0} + f^{ab}_{e}J^{d}_{-1}J^{e}_{1}J^{c}_{0} + f^{ac}_{e}J^{d}_{-1}J^{b}_{0}J^{e}_{1}\right) + k K_{db}K^{ad}J^{b}_{0}J^{c}_{0}\\
            &+ j_{1,1}f^{c}_{db} \left( f^{ad}_{e} J^{e}_{0}J^{b}_{0} + f^{ab}_{e} J^{d}_{-1} J^{e}_{1} \right) + kj_{1,1}f^{c}_{db}K^{ad}J^{b}_{0}\\
            &-2 j^{2}_{1,1} f^{ac}_{e} J^{e}_{0}-2 k j^{2}_{1,1}K^{ac}\\
        =& (2+k+2j_{1,1})J^{a}J^{c}_{0} + (2j_{1,1} + kj_{1,1}+2j^{2}_{1,1})f^{ca}_{e} - K^{ac}j_{1,1}(K_{eb}J^{e}_{0}J^{b}_{0}-2k j_{1,1})\\
            &+ K_{db}\left(f^{ab}_{e}J^{d}_{-1}J^{c}_{0} + f^{ac}_{e}J^{d}_{-1}J^{b}_{0} \right)J^{e}_{1}
             + \left(K_{db}f^{ab}_{h}f^{hc}_{e} + j_{1,1}f^{c}_{db}f^{ab}_{e}\right)J^{d}_{-1}J^{e}_{1}
    \end{aligned}
\end{equation}
Hence 
\begin{equation}
    J^{a}_{1} \hat{N}^{c}_{-1} \ket{j,m} = 0.
\end{equation}
\subsection{Subrepresentation}
The states $\hat{N}^{+}_{-1}\ket{j,m-1}$, $\hat{N}^{0}_{-1}\ket{j,m}$, and $\hat{N}^{-}_{-1} \ket{j,m+1}$ 
all have the same $J^{0}_{0}$ eigenvalue, namely $m$. Since the subrepresentation generated by the null vector is 
expected to be irreducible, these states should be linearly dependent. These states can be expanded in the basis of 
$J^{a}_{-1}\ket{j_{1,1},m}$. For simplicity, we omit the subscript and write $j$ for $j_{1,1}$.
\begin{equation}
    \begin{aligned}
        \hat{N}^{+}_{-1} \ket{j,m-1} = & ((j-m+1)(j+m) +2 j (m-1) -2 j^2) \, J^{+}_{-1}  \ket{j,m-1} \\
        & + (2(j-m+1)m-2j(j-m+1)) \, J^{0}_{-1}  \ket{j,m} \\
        & + (j-m+1)(j-m) \, J^{-}_{-1} \ket{j,m+1}\\
        = & (m-j)(j-m+1)\left( J^{+}_{-1} \ket{j,m-1} + 2 J^{0}_{-1} \ket{j,m} - J^{-1}_{-1} \ket{j,m+1} \right)\\
        \equiv  & -(j-m)(j-m+1) N_{-1} \ket{j,m}
    \end{aligned}
\end{equation}
Similarly, the other two states are 
\begin{equation}
    \begin{aligned}
        \hat{N}^{0}_{-1} \ket{j,m} = & (m(j+m) - j (j+m) ) \, J^{+}_{-1}  \ket{j,m-1} \\
        & + (2m^2 - 2j^2) \, J^{0}_{-1}  \ket{j,m} \\
        & + (m(j-m)+2(j-m)) \, J^{-}_{-1} \ket{j,m+1}\\
        = & (m-j)(m+j)\left( J^{+}_{-1} \ket{j,m-1} + 2 J^{0}_{-1} \ket{j,m} - J^{-1}_{-1} \ket{j,m+1} \right)\\
        = & -(j-m)(m+j) N_{-1} \ket{j,m}
    \end{aligned}
\end{equation}
\begin{equation}
    \begin{aligned}
        \hat{N}^{-}_{-1} \ket{j,m+1} = & (j+m+1)(j+m) \, J^{+}_{-1}  \ket{j,m-1} \\
        & + ((j+m+1)m+2j(j+m+1)) \, J^{0}_{-1}  \ket{j,m} \\
        & + ((j+m+1)(j-m)-2j(m+1)-2j^{2}) \, J^{-}_{-1} \ket{j,m+1}\\
        = & (j+m)(j+m+1)\left( J^{+}_{-1} \ket{j,m-1} + 2 J^{0}_{-1} \ket{j,m} - J^{-1}_{-1} \ket{j,m+1} \right)\\
        = & (j+m)(j+m+1) N_{-1} \ket{j,m}
    \end{aligned}
\end{equation}
Hence all these three states are propotional to each other. 
\par To find the spin of the subrepresentation, we first calculate the eigenstates of the Casimir operator.
The commutation relation of Casimir operator $C = K_{ab} J^{a}_{0} J^{b}_{0}$ with $J^{c}_{-1}$ is 
\begin{equation}
    \begin{aligned}
        \left[C,J^{c}_{-1}\right] = & K_{ab}\left[J^{a}_{0}J^{b}_{0},J^{c}_{-1}\right]\\
        =& K_{ab}f^{ac}_{d}J^{d}_{-1}J^{b}_{0} + K_{ab}f^{bc}_{d}J^{a}_{0}J^{d}_{-1}\\
        =& -f^{c}_{bd} J^{d}_{-1}J^{b}_{0} - f^{c}_{bd}J^{b}_{0}J^{d}_{-1}
    \end{aligned}
\end{equation}
The action of $C_{2}$ on basis $J^{a}_{-1} \ket{j,m}$ can be writen as the following matrix:
\begin{eqnarray}
    \begin{aligned}
        C_{2} 
    \begin{pmatrix}
    J^{+}_{-1} \ket{j,m-1}\\
    J^{0}_{-1} \ket{j,m}\\
    J^{-}_{-1} \ket{j,m+1}
    \end{pmatrix}
    = \begin{pmatrix}
        4m+2j(j+1) & -2 (j+m) & 0\\
         -4 (j-m+1) & 4 + 2j(j+1) & 4(j+m+1)\\
        0 & 2 (j-m)& -4m + 2j(j+1)
    \end{pmatrix}
    \begin{pmatrix}
        J^{+}_{-1} \ket{j,m-1}\\
        J^{0}_{-1} \ket{j,m}\\
        J^{-}_{-1} \ket{j,m+1}
    \end{pmatrix}
    \end{aligned}
\end{eqnarray}
After diagonalization, we find the eigenvectors and the corresponding eigenvalues are 
\begin{equation}
    \left\{
        \begin{aligned}
            j+1 &: -\frac{j+m}{j-m+1} J^{+}_{-1} \ket{j,m-1} + 2 J^{0}_{-1} \ket{j,m} + \frac{j-m}{j+m+1} J^{-}_{-1} \ket{j,m+1}\\
            j &: \frac{j+m}{m} J^{+}_{-1} \ket{j,m-1} + 2 J^{0}_{-1} \ket{j,m} + \frac{j-m}{m} J^{-}_{-1} \ket{j,m+1}\\
            j-1 &: J^{+}_{-1} \ket{j,m-1} + 2 J^{0}_{-1} \ket{j,m} - J^{-}_{-1} \ket{j,m+1}
        \end{aligned}
    \right.
\end{equation}
This is analogous to the tensor product of irreps of $\mathfrak{sl}_{2}$
\begin{equation}
    \mathrm{Adj} \otimes R_{j} = R_{j-1} \oplus R_{j} \oplus R_{j-1}.
\end{equation}
We find that $N_{-1}\ket{j,m}$ is exactly the eigenvector corresponding to $j-1$. Hence the subrepresentation has spin $j_{1,1}-1 = j_{1,-1}$.

\printbibliography 

\end{document}
