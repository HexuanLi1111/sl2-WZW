@ -1,59 +0,0 @@
\documentclass[10pt,a4paper]{article}
\usepackage[top=2.3cm,bottom=1.2cm,left=1.5cm,right=1.5cm]{geometry}% 设置 book 格式的页边距
\usepackage{indentfirst} %首行缩进宏包
\setlength{\parindent}{2em}
\usepackage{multicol}
\usepackage{setspace}
\usepackage{bm} %\bm 命令在数学环境下加粗符号, 可以用于标记矢量
\usepackage{pifont}
\usepackage{float}
\usepackage{cite}
\usepackage{amssymb}
\usepackage[tbtags]{amsmath}
\usepackage{amsmath}
\usepackage{tikz}
\numberwithin{equation}{section}
\renewcommand{\baselinestretch}{1.5} % 调整行间距, 推荐 1.5 倍行间距.

\newcommand{\ket}[1]{\left| #1 \right\rangle}
\newcommand{\vev}[1]{\left< #1 \right>}


\begin{document}
\title{Transformation between m and x basis}
\maketitle

The transformation from x-basis to $\mu$-basis is 
\begin{equation}
    \phi^{j}_{\mu} = \int \mathrm{d} x \, \mathrm{e}^{- \mu x} \phi^{j}_{x}.
\end{equation}

Consider the D series representation $\hat{D}^{j,+}$. The lowest weight state $\ket{j, -j}$ corresponds to $\mu$-basis field 
\begin{equation}
    \left.\phi^{j}_{\mu}\right|_{\mu = 0} = \int \mathrm{d} x \, \phi^{j}_{x}.
\end{equation}
This can be examined in $\mu$-basis: 
\begin{equation}
    J^{-}_{0} \phi^{j}_{ 0} = \left.\mu \phi^{j}_{\mu}\right|_{\mu = 0} = 0,
\end{equation}
or in x-basis:
\begin{equation}
    J^{-}_{0} \int \mathrm{d} x \, \phi^{j}_{x} = \int \mathrm{d} x \, - \partial_{x} \phi^{j}_{x} = 0.
\end{equation}

However, this will lead to a problem. The eigenvalue of $J^{0}_{0}$ on $\ket{j, -j}$ shoule be $-j$. On the other hand, we have 
\begin{align}
    J^{0}_{0} \phi^{j}_{\mu = 0} & = \int \mathrm{d} x \, J^{0}_{0} \phi^{j}_{x} \\
                                 & = \int \mathrm{d} x \, (x\partial_{x} - j) \phi^{j}_{x}\\
                                 & = (1-j) \phi^{j}_{\mu = 0} \neq -j \phi^{j}_{\mu = 0}.
\end{align}

\textcolor{red}{The complex conjugate of $j$ rep. is $-j-1$, which keeps the quadratic Casimir invariant.}

\end{document}